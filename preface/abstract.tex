\prefacesection{Abstract}
Predicting future environmental information is one of the most elusive and long-standing challenges in Artificial Intelligence. This report tries to tackle the problem of predicting the Arctic sea ice: how to implement multiple Machine Learning algorithms. On the one hand, we think that Supervised Learning algorithm can predict the Arctic sea ice level. On the other hand, if we want to build an optimal predictor to ensure that our predictions on the Arctic sea ice are useful to the further future research, we would need an efficient and concise strategy to analyse the performance of different Supervised Learning algorithms. Besides, we need more strategies to analyse different factors related to the level of sea ice.

In this project, we firstly focus on Linear Regression, Penalised linear regression, Penalised Polynomial Regression, Random Forest, and Neural Networks: the most common and famous Supervised Learning algorithms by far. Then, according to the testing results, we will choose a better algorithm after testing the performance to be the algorithm that predicts the Arctic sea ice, both in a normal situation and the selected special situations.

This report consists of two parts. In the first part, we aim to understand the advantages of predicting sea ice of the Arctic and previous research on applying all kinds of algorithms on the related topic, especially predicting the Arctic sea ice or the Antarctic ice sheets. Then we present our efforts at implementing effective Supervised Learning algorithms. 

In the second part of this project, we first get our performance results with different Supervised Learning algorithms by K-fold method. This is the so-called now-forecasting. Then, based on the performance results, we predict the Arctic sea ice for the further ten years with Random Forest with different specific scenarios.