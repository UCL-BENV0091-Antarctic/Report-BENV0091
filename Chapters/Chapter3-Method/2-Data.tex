\section{Data} %3.1
\subsection{Data Sources} %3.1.1
For the dependent variable in the project, data for Arctic sea ice could be found in National Snow \& Ice Data Centre (NSIDC). For independent variables, data for global CO2 content and Arctic ozone hole area are made available to the public via the National Aeronautics and Space Administration (NASA) website. Data for the global population were accessed through the Our World in Data website. In addition, data for the different temperatures and rainfall and average daylight of Arctic are provided by National Oceanic and Atmospheric Administration (NOAA) and Weather Atlas website respectively. Furthermore, the data for the global GDP could be found in the world bank website.

All independent variables data were based on monthly records during the period from January 1980 to October 2020. In the process of collecting the data, the group found that some data such as population and GDP contain only annual data set. Therefore, it is necessary to convert these data into monthly format applying Mean Reversion. Moreover, there is a small amount of data missing from the original data (due to the failure of instrumentation). Thus, Mean Reversion can be applied to supplement missing data as well.

\subsection{Data Normalisation} %3.1.2
Normalisation was applied on the whole data sets for lower noise generated by the different magnitude of data during the training process \cite{bessembinder1995mean}. For the process of normalisation, the minimum value in the data set was placed to 0, while the maximum was placed to 1. For processing the whole data set, Function \ref{eq:NV} was applied.
\begin{eqnarray}
    {\text{Normalised Value}} = {\frac {{\text{Value}}-{\text{Minimum Value}} }{{\text{Maximum Value}}-{\text{Minimum Value}}}}
    \label{eq:NV}
\end{eqnarray}