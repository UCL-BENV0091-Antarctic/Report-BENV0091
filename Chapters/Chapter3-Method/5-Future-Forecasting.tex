\section{Future Forecasting Methodology} %3.6
\subsection{Normal Situation} %3.6.1
\label{sec:normal-predic}
In order to predict the Arctic sea ice extent variation in the next 10 years, the existing features will be processed in three ways to obtain the value of each feature from the year of 2021 to 2030.
\begin{enumerate}[(a)]
    \item 
        \textbf{Pure Periodical Feature (Daylight, Rainfall)}. Theses features are monthly-average values over years since 20th century. They would be applied on the predictions for the next 10 years.
        
    \item
        \textbf{Data with Accessible Future Scenario (Population, GDP, CO2)}. Since the future variation scenarios are given, the monthly record in scenarios would be applied directly.
    \item
        \textbf{Other data (Ozone, Temperature related features)}. As no future variation scenario was found, the values of future 10 years would be generated by applying existing trends via linear regression.
\end{enumerate}

The way of generating future data is based entirely on recent trends, so the predicted result would be treated as \sys{Normal Situation}. It would be compared as criteria against forecast based on \sys{Special Situations} which would be discussed in the following part.


\subsection{Special Situation} %3.6.2
The \sys{Special Situations} was defined as the prediction of the Arctic sea ice variation under the circumstance that the fluctuation of a feature does not follow the trend of previous years. Firstly, two feature that has the most significant impact on extent variation would be obtained by variable importance analysis as key factors. For the key factors, two different scenarios would be generated, correspondingly representing sharper and smoother variation comparing with \sys{Normal Situation}.