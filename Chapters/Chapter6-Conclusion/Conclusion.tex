% Chapter 6: Conclusion

In this report, we gave readers a thorough overview of predicting the Arctic sea ice in Supervised Learning: the foundations (\sys{Part I}) and the scenario (\sys{Part II}), as well as how we contributed to the analytical strategy to connect the theories and testings.

In Chapter~\ref{Chapter2:Review}, we first explain how important the sea ice to the climate and environment research, especially the global warm. Then, we walked through the overview of the field of environmental risks in terms of multiple kinds of algorithms, including predicting sea ice level using statistical models, predicting sea ice level using Machine Learning models, and predicting sea ice level using non-statistical methods or non-machine learning methods.

In Chapter~\ref{Chapter3:Method}, we covered the analytical strategies of our project. We explained the features we used and where we can obtain these data. Model accuracy is one of the most important steps in our project. We used R-squared and mean square error (MSE) to analyse the accuracy of our Supervised Learning algorithms. K-fold cross-validation was used in this project since small data were used to predict the sea ice level in this project. Then, in mathematically, we introduced the cost function for Linear Regression algorithm and explained how it worked. On the basis of Linear Regression, we introduced penalised linear regression by adding a penalised term. Furthermore, penalised polynomial regression can be derived by applying higher-order parameters on features. Last but not least, we introduced the famous Random Forest and Neural Networks. 

In \sys{Part II}, the key questions we want to answer are: which factor(s) is/are the most important factor to the change of the sea ice, and more importantly, how the sea ice will be melt if the situations go worse?

In Chapter~\ref{Chapter4:Now-Results}, we gave readers clear visualisations on correlation relationships and the fitting diagram of each Supervised Learning model. At the end of chapter, in Section \ref{sec:Comparison}, we summarised that Random Forest had the best fitting performance in our case.

In Chapter~\ref{Chapter5:Future-Results}, we designed three specific scenarios for our future prediction. The first one is \sys{normal situation} where all factors/features are followed the current trends. Then we designed two \sys{special situations} scenarios for the future prediction. We first selected two factors that influence the sea ice level the most, then these factors are applied on \sys{special situations} respectively. In a result, for the worst situation, the extent of the Arctic sea ice will lose around 1\% more compared to the \sys{normal situation}. It is time for policy-makers to take action to control the global temperature and reduce the excessive greenhouse effect.

We am really excited about the progress that has been made for this semester. At the same time, we also deeply believe that there is still a long way to go towards predicting the Arctic sea ice or other environmental information, and we are still facing challenges and a lot of open questions that need to be addressed in the future. For instance, we need spend more time on more scenario analysis. For instance, we need to find out what the "high-level scenario" means for Ozone. Is increasing 2.5\% in 10 years for "high-level scenario" in our case reasonable? Secondary, we need to find our more possible features. Thirdly, a large and tidy data set is important for the scientific research since big data are helpful to show the advantage of Neural Networks or other Artificial Intelligence algorithms.