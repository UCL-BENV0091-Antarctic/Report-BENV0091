% Chapter 2: Literature Review

\section{Why We Choose to Predict Sea Ice}
The change of Arctic sea ice is a sensitive indicator of climate change. The rate of Arctic sea ice disappearance exceeds even the most pessimistic climate model predictions. The current Arctic summer ice conditions are 30 years earlier than model predictions on average, and seasonal ice-free conditions are happened early \cite{stroeve_frei_mccreight_ghatak_2008}. Therefore, the prediction of sea ice is essential for understanding the future Arctic environment and global changes. Global warming has led to a reduction in sea ice and aggravated the deterioration of the Arctic environment, while the reduction in sea ice, in turn, has accelerated global warming. Increasing concentrations of greenhouse gases are playing an increasingly important role in the disappearance of the Arctic ice cap \cite{kim2019satellite}. Many studies have also linked the loss of sea ice to atmospheric circulation patterns.

\section{How to Predict Sea Ice Level Using Statistical Models}
There are many statistical models that study the relationship between the Arctic sea ice concentration and climate factors. Stroeve et al. pioneered the use of singular value decomposition SVD, empirical orthogonal function EOF and other multivariate analysis techniques, focusing on the relationship between the decline of sea ice concentration in summer and winter and the warming trend and AO driving \cite{stroeve_frei_mccreight_ghatak_2008}. TIvy et al. used the multivariate analysis technique of Code Correlation Analysis (CCA) to compare the SIC value of Hudson Bay in July from 1971 to 2005 by comparing sea surface temperature, position altitude, sea level pressure, and regional surface air temperature. Among them, in the 6-month forecast period, the forecast results are the most accurate. Surface temperature in autumn is the most influential predictor \cite{tivy2011origins}. Ahn et al. introduced the automatic regression integrated moving average (ARIMA) method for the first time into the sea ice concentration statistical model, using seven climatic factors (skin temperature, sea surface temperature, total column liquid water, total column water vapour, and instantaneous moisture flux). It is better at predicting large data sets than the single equation of the ordinary minimum hours (OLS) regression method and has higher accuracy. The average improvement of RMSE is 0.076 \cite{ahn2014statistical}. The forward stepwise regression model is used to predict summer sea ice conditions several months in advance based on four predictors which are winter multi-year combined concentration, total ice concentration in spring, North Atlantic Oscillation Index and East Atlantic Index where $\text{R}{^2}$ value is 90.7\% and $\text{MAE}$ value is 34 \cite{drobot2002practical}. In 2003, it was expanded with multiple linear regression models \cite{drobot2003long}. Finally, in 2007, on this basis, Drobot and others created a multiple linear regression model (MLR) to predict the annual minimum Arctic sea ice range for the monthly interval from February to August. The forecast data is based on the average monthly weighted index of sea ice concentration (WIC), surface skin temperature (WST), surface illuminance (WAL) and surface long-pass volume (WDL). Each MLR model is better than the climatology model \cite{drobot2006long}, but due to insufficient statistical time series modelling, the performance of machine learning models is often better than statistical models.

\section{How to Predict Sea Ice Level Using Machine Learning Models}
Machine learning models are often used to predict Arctic sea ice concentration and Arctic sea ice classification. Chi et al. used a large Arctic sea ice dataset to train a neural network to predict the Arctic sea ice concentration \cite{chi2017prediction}. The neural network prediction results that use long and short-term memory (LSTM) are better than traditional autoregressive (AR) models, and can successfully adapt to long-term data sets. The average monthly forecast error is less than 9\%, but the predictability in summer is low \cite{chi2017prediction}. Choi et al. used artificial neural networks (ANN) to make short-term predictions of the Arctic Sea Ice Concentration (SIC) \cite{choi2019artificial}. Using global SIC data for training will result in higher prediction accuracy than using only Arctic SIC data for training \cite{choi2019artificial}. Shu et al. proposed an object-based random forest (ORF) to classify the ice map types of the Arctic, and the overall classification accuracy was 90.1\%. When providing surface-related values of sea ice density and snow cover, the estimation of ice thickness can be improved \cite{shu2020discrimination}.


\section{How to Predict Sea Ice Level Using Non-Statistical Methods or Non-Machine Learning Methods}
The field of sea ice prediction is very extensive. Since 1979, the satellite-based multi-channel passive microwave imaging system has continuously monitored the Arctic sea ice concentration. Common monitoring systems are Scanning Multichannel Microwave Radiometer (SMMR), Special Sensor Microwave/Imager (SSM/I) and Advanced Microblog Scanning Radiometer (AMSR) \cite{fennig2020fundamental}. Numerical models predict interactions based on physical equations. In the short term, The prediction is usually better than the statistical model. However, it is difficult and expensive to obtain data by physical models \cite{chi2017prediction}. Sea ice retrieval algorithms usually process satellite data, and various sea ice parameters have been determined, such as age, concentration, range, thickness, etc. \cite{chi2017prediction}, but statistical models are usually better than dynamic models \cite{tivy2011origins}.